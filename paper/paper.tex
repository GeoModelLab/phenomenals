% Options for packages loaded elsewhere
\PassOptionsToPackage{unicode}{hyperref}
\PassOptionsToPackage{hyphens}{url}
\PassOptionsToPackage{dvipsnames,svgnames,x11names}{xcolor}
%
\documentclass[
  letterpaper,
  DIV=11,
  numbers=noendperiod]{scrartcl}

\usepackage{amsmath,amssymb}
\usepackage{iftex}
\ifPDFTeX
  \usepackage[T1]{fontenc}
  \usepackage[utf8]{inputenc}
  \usepackage{textcomp} % provide euro and other symbols
\else % if luatex or xetex
  \usepackage{unicode-math}
  \defaultfontfeatures{Scale=MatchLowercase}
  \defaultfontfeatures[\rmfamily]{Ligatures=TeX,Scale=1}
\fi
\usepackage{lmodern}
\ifPDFTeX\else  
    % xetex/luatex font selection
\fi
% Use upquote if available, for straight quotes in verbatim environments
\IfFileExists{upquote.sty}{\usepackage{upquote}}{}
\IfFileExists{microtype.sty}{% use microtype if available
  \usepackage[]{microtype}
  \UseMicrotypeSet[protrusion]{basicmath} % disable protrusion for tt fonts
}{}
\makeatletter
\@ifundefined{KOMAClassName}{% if non-KOMA class
  \IfFileExists{parskip.sty}{%
    \usepackage{parskip}
  }{% else
    \setlength{\parindent}{0pt}
    \setlength{\parskip}{6pt plus 2pt minus 1pt}}
}{% if KOMA class
  \KOMAoptions{parskip=half}}
\makeatother
\usepackage{xcolor}
\setlength{\emergencystretch}{3em} % prevent overfull lines
\setcounter{secnumdepth}{5}
% Make \paragraph and \subparagraph free-standing
\makeatletter
\ifx\paragraph\undefined\else
  \let\oldparagraph\paragraph
  \renewcommand{\paragraph}{
    \@ifstar
      \xxxParagraphStar
      \xxxParagraphNoStar
  }
  \newcommand{\xxxParagraphStar}[1]{\oldparagraph*{#1}\mbox{}}
  \newcommand{\xxxParagraphNoStar}[1]{\oldparagraph{#1}\mbox{}}
\fi
\ifx\subparagraph\undefined\else
  \let\oldsubparagraph\subparagraph
  \renewcommand{\subparagraph}{
    \@ifstar
      \xxxSubParagraphStar
      \xxxSubParagraphNoStar
  }
  \newcommand{\xxxSubParagraphStar}[1]{\oldsubparagraph*{#1}\mbox{}}
  \newcommand{\xxxSubParagraphNoStar}[1]{\oldsubparagraph{#1}\mbox{}}
\fi
\makeatother


\providecommand{\tightlist}{%
  \setlength{\itemsep}{0pt}\setlength{\parskip}{0pt}}\usepackage{longtable,booktabs,array}
\usepackage{calc} % for calculating minipage widths
% Correct order of tables after \paragraph or \subparagraph
\usepackage{etoolbox}
\makeatletter
\patchcmd\longtable{\par}{\if@noskipsec\mbox{}\fi\par}{}{}
\makeatother
% Allow footnotes in longtable head/foot
\IfFileExists{footnotehyper.sty}{\usepackage{footnotehyper}}{\usepackage{footnote}}
\makesavenoteenv{longtable}
\usepackage{graphicx}
\makeatletter
\newsavebox\pandoc@box
\newcommand*\pandocbounded[1]{% scales image to fit in text height/width
  \sbox\pandoc@box{#1}%
  \Gscale@div\@tempa{\textheight}{\dimexpr\ht\pandoc@box+\dp\pandoc@box\relax}%
  \Gscale@div\@tempb{\linewidth}{\wd\pandoc@box}%
  \ifdim\@tempb\p@<\@tempa\p@\let\@tempa\@tempb\fi% select the smaller of both
  \ifdim\@tempa\p@<\p@\scalebox{\@tempa}{\usebox\pandoc@box}%
  \else\usebox{\pandoc@box}%
  \fi%
}
% Set default figure placement to htbp
\def\fps@figure{htbp}
\makeatother

\KOMAoption{captions}{tableheading}
\makeatletter
\@ifpackageloaded{caption}{}{\usepackage{caption}}
\AtBeginDocument{%
\ifdefined\contentsname
  \renewcommand*\contentsname{Table of contents}
\else
  \newcommand\contentsname{Table of contents}
\fi
\ifdefined\listfigurename
  \renewcommand*\listfigurename{List of Figures}
\else
  \newcommand\listfigurename{List of Figures}
\fi
\ifdefined\listtablename
  \renewcommand*\listtablename{List of Tables}
\else
  \newcommand\listtablename{List of Tables}
\fi
\ifdefined\figurename
  \renewcommand*\figurename{Figure}
\else
  \newcommand\figurename{Figure}
\fi
\ifdefined\tablename
  \renewcommand*\tablename{Table}
\else
  \newcommand\tablename{Table}
\fi
}
\@ifpackageloaded{float}{}{\usepackage{float}}
\floatstyle{ruled}
\@ifundefined{c@chapter}{\newfloat{codelisting}{h}{lop}}{\newfloat{codelisting}{h}{lop}[chapter]}
\floatname{codelisting}{Listing}
\newcommand*\listoflistings{\listof{codelisting}{List of Listings}}
\makeatother
\makeatletter
\makeatother
\makeatletter
\@ifpackageloaded{caption}{}{\usepackage{caption}}
\@ifpackageloaded{subcaption}{}{\usepackage{subcaption}}
\makeatother

\usepackage{bookmark}

\IfFileExists{xurl.sty}{\usepackage{xurl}}{} % add URL line breaks if available
\urlstyle{same} % disable monospaced font for URLs
\hypersetup{
  pdftitle={The PhenoMeNals Approach for Grapevine Yield and Quality Forecasting},
  pdfauthor={Simone Bregaglio, Sofia Bajocco},
  colorlinks=true,
  linkcolor={blue},
  filecolor={Maroon},
  citecolor={Blue},
  urlcolor={Blue},
  pdfcreator={LaTeX via pandoc}}


\title{The PhenoMeNals Approach for Grapevine Yield and Quality
Forecasting}
\author{Simone Bregaglio, Sofia Bajocco}
\date{}

\begin{document}
\maketitle

\renewcommand*\contentsname{Table of contents}
{
\hypersetup{linkcolor=}
\setcounter{tocdepth}{3}
\tableofcontents
}

\section{Introduction}\label{introduction}

Grapevine (\emph{Vitis vinifera} L.) is one of the most widely
cultivated fruit crops globally, with high economic value and strong
sensitivity to climatic variability. Accurate forecasting of yield and
quality is essential for decision-making in viticultural management,
allowing producers to anticipate harvest volume, allocate labor and
inputs efficiently, and adapt wine production strategies accordingly
{[}@fraga2019; @vanleeuwen2022{]}. The need for robust forecasting tools
is particularly pressing under current climate change scenarios, which
are increasing the frequency of extreme events, phenological shifts, and
yield variability in perennial crops {[}@chuine2017; @santos2020{]}.

Various modeling approaches have been proposed to predict grapevine
yield and quality. These include empirical models based on historical
correlations with meteorological variables {[}@goulet2019{]},
process-based models incorporating phenological and physiological
processes {[}@keller2010{]}, and machine learning approaches trained on
large multivariate datasets {[}@oliveira2021{]}. While each class of
models has shown utility under specific conditions, several limitations
persist. Empirical models often lack generalizability and physiological
interpretability. Mechanistic models are constrained by data
requirements and calibration complexity. Data-driven methods can capture
nonlinear relationships but frequently act as black boxes, with limited
transparency and interpretability for agronomic decision-making.

Moreover, most forecasting models treat phenology in a static or
coarse-grained way---e.g., as discrete BBCH stages---and tend to
aggregate agrometeorological predictors over fixed calendar or
phenophase periods {[}@grifoni2006{]}. This limits their capacity to
capture the dynamic, continuous interaction between plant development
and environmental stressors throughout the season. Additionally, a
growing body of evidence indicates that grapevine yield and quality are
influenced not only by the climatic conditions of the current growing
season but also by those of the previous year {[}@santesteban2011;
@greer2013; @keller2010{]}. These carry-over effects, mediated by
reserve accumulation, bud differentiation, and adaptation mechanisms,
are rarely incorporated explicitly into forecasting models.

To our knowledge, no previous approach has operationalized this concept
of \emph{phenological memory} using dynamic, time-resolved
agrometeorological indicators in a predictive framework. In particular,
no existing model applies rolling correlation analyses across the
phenological continuum---spanning dormancy and active growth---and
accumulates statistically weighted signals from both current and
previous years as predictors for yield or quality.

In this study, we introduce \textbf{PhenoMeNals (Phenology Memory
Signals)}, an innovative, open-source R package and forecasting approach
for grapevine production modeling. The PhenoMeNals method integrates:

\begin{enumerate}
\def\labelenumi{\arabic{enumi}.}
\item
  A self-calibrating phenological model simulating the dormancy
  (0--100\%) and growing season (100--200\%) using chill/anti-chill and
  forcing functions aligned with BBCH observations;
\item
  Eight agrometeorological indicators (phenomenals) reflecting key
  physiological and pathological risk drivers (e.g., thermal response,
  water stress, heat/cold extremes, disease-conducive conditions);
\item
  A novel use of rolling Pearson correlations between these indicators
  and the target variable, computed over current and prior years,
  normalized via sigmoid scaling and statistical significance;
\item
  A final multiple regression model based on selected predictors,
  evaluated for collinearity, significance, and temporal stability.
\end{enumerate}

The overall goal of this work is to advance forecasting methodology for
perennial crops by incorporating climate-driven memory effects into a
transparent and reproducible framework. The approach is implemented as a
fully open R package available on GitHub, supporting further
applications across cultivars and regions.




\end{document}
